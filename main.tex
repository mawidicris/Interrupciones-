\documentclass{article}
\usepackage[utf8]{inputenc}

\title{Proycyo}
\author{mawidicris }
\date{March 2020}

\usepackage{natbib}
\usepackage{graphicx}

\begin{document}
\begin{center}
\includegraphics[scale=0.090]{Escudo-UdeA.svg.png}
\end{center}
\vspace{50pt}
\begin{center}
\bf{\sc\Large 'Interrupciones a nivel del procesador'}\\
\end{center}
\vspace{50pt}
\begin{center}
\begin{center}
\bf{\sc\large Por:}\\
\end{center}
\bf{\sc\large Cristian Daniel Padrón Hernández}\\
\end{center}
\begin{center}

\bf{\sc\large C.C: 1152717544}\\
\end{center}
\vspace{50pt}
\begin{center}
\bf{\sc\large Facultad de ingeniería}\\
\end{center}
\begin{center}
\bf{\sc\large Medellín}
\end{center}
\begin{center}
\bf{\sc\large 2020}\\
\end{center}\





\newpage
\Large
\section{¿Qué es una interrupción en el contexto de los microprocesadores?}

Tambien es conocido como corrupción del hardware o petición de interrupción, es un proceso en el cual se le envía una señal al procesador del equipo para ordenar la interrupción del proceso que se está realizando actualmente y proceda a realizar otro que se ha especificado con anterioridad. 


\newline


\newpage

\begin{thebibliography}{X}
\bibitem{Baz} \textsc{John T. Baldwin} y \textsc{Olivier Lessmann},
\textit{What is Russell's paradox?}, Recuperado de: https://www.scientificamerican.com/article/what-is-russells-paradox/,
1998
\bibitem{Baz}
\textit{Teoremas de incompletitud de Gödel}, Recuperado de: https://es.wikipedia.org/wiki/Teoremas_de_incompletitud_de_G%C3%B6del

\bibitem{Baz} \textsc{Joseph Warren Dauben} 
\textit{GEORG CANTOR: His Mathematics and Philosophy of the Infinite}, Recuperado de: https://math.dartmouth.edu/~matc/Readers/HowManyAngels/Cantor/Cantor.html, 1979

\end{thebibliography}
\end{document}
